\documentclass{article}

\usepackage[colorlinks=true, linkcolor=blue, citecolor=blue]{hyperref}
\usepackage{cite}

\begin{document}

\subsection*{Abstract}
The ALICE experiment consists of a central barrel and a forward muon
spectrometer~\cite{Aamodt:2008zz}. Additional smaller detectors for global event
characterization and triggering are located at small angles outside of
the central barrel. Such a geometry allows the investigation of many
properties of diffractive reactions at hadron colliders for example the
measurement of single and double diffractive dissociation cross-sections
and the study of central exclusive production (CEP). Central diffractive
events are defined experimentally by hits in the central barrel and no
activity outside of it creating an activity gap in the
rapidity observable~\cite{Schicker:2014wvk}.

Studying \emph{Pythia-8} simulations of these processes shows a drastic
reduction of non-diffractive events (background) by enforcing the
rapidity gap condition. The remaining background is largely composed of
partially reconstructed CEP events, so called feed down events. Often
feed down events are accompanied by neutral particles which are not
detected. This missing mass and momentum leads to a shift of the
invariant mass spectrum to lower masses. This work aims at applying machine
learning methods for background suppression of CEP events. 

The measured variables \emph{e.g.} the four-momentum of particles, energy loss in the 
detectors, deduced kinematic quantities, and global event characteristics are in general correlated. 
To obtain a maximal separation of signal and background it is necessary to treat these observables in a
fully multivariate way.

In this talk I will give an introduction to the concepts of machine learning and discuss its application to the 
analysis of central exclusive production events with ALICE.

% A reason to apply statistical training multivariate methods is simply
% the lack of knowledge about the mathematical dependence of the quantity
% of interest on the relevant measured variables.

\bibliographystyle{plain}
\bibliography{refs}
 


\end{document}
