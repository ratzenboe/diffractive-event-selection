\documentclass[]{article}
\usepackage{tikz}

\makeatletter
\newcount\dirtree@lvl
\newcount\dirtree@plvl
\newcount\dirtree@clvl
\def\dirtree@growth{%
  \ifnum\tikznumberofcurrentchild=1\relax
  \global\advance\dirtree@plvl by 1
  \expandafter\xdef\csname dirtree@p@\the\dirtree@plvl\endcsname{\the\dirtree@lvl}
  \fi
  \global\advance\dirtree@lvl by 1\relax
  \dirtree@clvl=\dirtree@lvl
  \advance\dirtree@clvl by -\csname dirtree@p@\the\dirtree@plvl\endcsname
  \pgf@xa=5mm\relax
  \pgf@ya=-5mm\relax
  \pgf@ya=\dirtree@clvl\pgf@ya
  \pgftransformshift{\pgfqpoint{\the\pgf@xa}{\the\pgf@ya}}%
  \ifnum\tikznumberofcurrentchild=\tikznumberofchildren
  \global\advance\dirtree@plvl by -1
  \fi
}

\tikzset{
  dirtree/.style={
    growth function=\dirtree@growth,
    every node/.style={anchor=north},
    every child node/.style={anchor=west},
    edge from parent path={(\tikzparentnode\tikzparentanchor) |- (\tikzchildnode\tikzchildanchor)}
  }
}
\usetikzlibrary{fit}
\newcommand\addvmargin[1]{
      \node[fit=(current bounding box),inner ysep=#1,inner xsep=0]{};
}

\makeatother
\begin{document}
\begin{center}
    \setlength{\tabcolsep}{9mm} % separator between columns
    \def\arraystretch{1.25} % vertical stretch factor
    \centering
    \begin{tabular}{c | c} 
        Decay & Occurance[\%] \\ 
        \hline
        \hline
        \begin{tikzpicture}[dirtree, baseline=(current bounding box.center)]\centering\node{$X$} child { node {$\pi^{0}$} child { node {$\pi^{+}$} } child { node {$\pi^{-}$} } } child { node {$\rho^{0}$} child { node {$2\gamma$} } } ;\addvmargin{1mm}\end{tikzpicture}  & 100.00\\ \hline

    \end{tabular}
\end{center}


\end{document}


% -------- Example decay chain --------------
%     \begin{tikzpicture}[dirtree]
%     \node {X} 
%         child { node {$\pi^{0}$}
%             child {node {$2\gamma$} }
%         }
%         child { node {$\pi^{+}$} }
%         child { node {$N$} }
%         child { node {$\bar{p}$} };
%         child { node {Bar}
%             child { node {foo} }
%             child { node {foo} }
%             child { node {foo} }
%             child { node {foo} }
%             child { node {bar} }
%             child { node {baz} }
%         }
%         child { node {Baz}
%             child { node {foo} }
%             child { node {bar} }
%             child { node {baz} }
%         };
%     \end{tikzpicture}
